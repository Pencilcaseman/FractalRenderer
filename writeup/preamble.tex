\usepackage{amsmath}
\usepackage{amssymb}
\usepackage{tikz}
\usepackage{pgfplots}
\usepackage{subfig}
\usepackage{physics}
\usepackage{algorithmicx}
\usepackage{algpseudocode}
\usepackage{listings}     
\usepackage{lstautogobble}  % Fix relative indenting
\usepackage{color}          % Code coloring
\usepackage{zi4}            % Nice font
\usepackage{hyperref}
\usepackage{yfonts}
\usepackage{relsize}
\usepackage{hyperref}
\usepackage{xcolor}
\usepackage{graphicx}
\usepackage{lipsum}
\usepackage{adjustbox}
\usepackage{flushend}
\usepackage{booktabs}
\usepackage{siunitx}
\usepackage{blindtext}
\usepackage{multicol}
\usepackage[breakable,
skins]{tcolorbox}
\usepackage[protrusion=true,
activate={true,nocompatibility},
final,
tracking=true,
kerning=true,
spacing=true,
factor=1100]{microtype}
\usepackage{fontawesome}
\usepackage{notoccite}
\usepackage[numbers,sort&compress]{natbib}
\usepackage{placeins}
\usepackage{longtable}
\usepackage{fancyhdr}
\usepackage{zref-totpages}
\usepackage{tablefootnote}

\usetikzlibrary{shapes, decorations, calc, arrows, arrows.meta}
\usetikzlibrary{3d,fit,backgrounds, decorations.text}
\usetikzlibrary{positioning, shapes.symbols}
\usetikzlibrary{decorations.pathreplacing, calligraphy}

\pgfplotsset{width=10cm, compat=1.18}
\usepgfplotslibrary{external}

% \tikzexternalize[prefix=cache/tikz]
\tikzset{>=latex}

\SetTracking{encoding={*}, shape=sc}{40}

\hypersetup{
	colorlinks,
	linkcolor={red!50!black},
	citecolor={blue!50!black},
	urlcolor={blue!80!black}
}

\lstdefinestyle{paperStyle1}{
	backgroundcolor=\color{backcolour},   
	commentstyle=\color{codegreen},
	keywordstyle=\color{magenta},
	numberstyle=\tiny\color{codegray},
	stringstyle=\color{codepurple},
	basicstyle=\ttfamily\footnotesize,
	breakatwhitespace=false,         
	breaklines=true,                 
	captionpos=b,                    
	keepspaces=true,                 
	numbers=left,                    
	numbersep=5pt,                  
	showspaces=false,                
	showstringspaces=false,
	showtabs=false,                  
	tabsize=2
}

% ================================================================================================
% From https://tex.stackexchange.com/questions/94902/how-can-i-use-line-numbers-with-leading-zeros
% Pads codeblock line numbers with zeros and aligns them correctly

\errorcontextlines=\maxdimen

\newcommand*{\boxeddecimalnum}[1]{\makebox[3em][r]{\decimalnum{#1}}}
\usepackage{refcount}
\usepackage{fmtcount}
\usepackage{intcalc}

\def\getnumdigitsaux #1{%
	\ifx#1\quark
	\expandafter\relax
	\else
	+1\expandafter\getnumdigitsaux
	\fi
}

\def\quark{\quark}
\newcommand\getnumdigits[1]{%
	\the\numexpr\getnumdigitsaux #1\quark
}

\newcounter{lstuniquenumber}
\makeatletter
\lst@AddToHook{Init}{\stepcounter{lstuniquenumber}\edef\lastlineincurrentlisting{\intcalcDec{\getrefnumber{lastlineinlisting\thelstuniquenumber}}}\padzeroes[\expandafter\getnumdigits\expandafter{\lastlineincurrentlisting}]} % at the start of each listing, count the listing, and try to restore the line number saved in the previous run
\lst@AddToHook{DeInit}{\label{lastlineinlisting\thelstuniquenumber}} % at the end of each listing, save one past the last line number
\makeatother

% ================================================================================================

\lstdefinestyle{paperStyle2}{
	autogobble,
	columns=fullflexible,
	showspaces=false,
	showtabs=false,
	breaklines=true,
	showstringspaces=false,
	breakatwhitespace=true,
	escapeinside={(*@}{@*)},
	identifierstyle=\color{pinkidentifiers},
	commentstyle=\color{greencomments},
	keywordstyle=\color{bluekeywords},
	stringstyle=\color{redstrings},
	% prebreak=\raisebox{0ex}[0ex][0ex]{\ensuremath{\hookleftarrow}},
	numberstyle=\color{graynumbers}\ttfamily\boxeddecimalnum,
	morekeywords={psrotate},
	basicstyle=\ttfamily\footnotesize,
	frame=1,
	numbers=left,
	framesep=22pt,
	tabsize=4,
	captionpos=b,
	xleftmargin=5.5ex
}

% \definecolor{bluekeywords}{rgb}{0.13, 0.13, 1}
% \definecolor{greencomments}{rgb}{0, 0.5, 0}
% \definecolor{redstrings}{rgb}{0.9, 0, 0}
% \definecolor{graynumbers}{rgb}{0.5, 0.5, 0.5}

\definecolor{pinkidentifiers}{rgb}{1.00, 0.40, 0.40}
\definecolor{bluekeywords}{rgb}{0.40, 0.40, 1.00}
\definecolor{greencomments}{rgb}{0.00, 0.50, 0.00}
\definecolor{redstrings}{rgb}{0.90, 0.00, 0.00}
\definecolor{graynumbers}{rgb}{0.50, 0.50, 0.50}

\setlength\columnsep{30pt} % This is the default columnsep for all pages

\lstset{style=paperStyle2}
\sisetup{load-configurations = abbreviations, binary-units = true}
\DeclareSIUnit\px{px}

% Rename "Contents" to "Summary"
\renewcommand*\contentsname{Summary}

% Register codeword as a command
\NewDocumentCommand{\codeword}{v}{%
	\texttt{\textcolor{darkgray}{#1}}%
}

\graphicspath{{images/}}

% Nice grey box
\newtcolorbox{greybox}[1]{colback=black!5!white,colframe=white!25!black,title={#1}}

% Red box
\newtcolorbox{redbox}[1]{colback=red!5!white,colframe=red!50!black,title={#1}}

% Blue box
\newtcolorbox{bluebox}[1]{colback=blue!5!white,colframe=blue!50!black,title={#1}}

% Green box
\newtcolorbox{greenbox}[1]{colback=green!5!white,colframe=green!50!black,title={#1}}

% Code box
\newtcolorbox{codebox}[1]{colback=blue!15!black,colframe=white!50!black,title={\faCode \quad #1}}

% Define a nice C++ typesetting
\newcommand\CPP{C\nolinebreak[4]\hspace{-.05em}\raisebox{.4ex}{\relsize{-3}{\textbf{++}}}}

% Tikz Style %
\usetikzlibrary{shapes,arrows}
\tikzstyle{thing} = [rectangle, minimum width=3cm, minimum height=1cm, text centered, draw=black, inner sep=3.5mm]
\tikzstyle{arrow} = [thick,->,>=stealth]

\makeatletter
\pgfarrowsdeclare{crow's foot}{crow's foot}
{
	\pgfarrowsleftextend{+-.5\pgflinewidth}%
	\pgfarrowsrightextend{+.5\pgflinewidth}%
}
{
	\pgfutil@tempdima=0.6pt%
	\pgfsetdash{}{+0pt}%
	\pgfsetmiterjoin%
	\pgfpathmoveto{\pgfqpoint{0pt}{-9\pgfutil@tempdima}}%
	\pgfpathlineto{\pgfqpoint{-13\pgfutil@tempdima}{0pt}}%
	\pgfpathlineto{\pgfqpoint{0pt}{9\pgfutil@tempdima}}%
	\pgfusepathqstroke%
}

\pgfarrowsdeclare{omany}{omany}
{
	\pgfarrowsleftextend{+-.5\pgflinewidth}%
	\pgfarrowsrightextend{+.5\pgflinewidth}%
}
{
	\pgfutil@tempdima=0.6pt%
	\pgfsetdash{}{+0pt}%
	\pgfsetmiterjoin%
	\pgfpathmoveto{\pgfqpoint{0pt}{-9\pgfutil@tempdima}}%
	\pgfpathlineto{\pgfqpoint{-13\pgfutil@tempdima}{0pt}}%
	\pgfpathlineto{\pgfqpoint{0pt}{9\pgfutil@tempdima}}%
	\pgfpathmoveto{\pgfqpoint{0\pgfutil@tempdima}{0\pgfutil@tempdima}}%  
	\pgfpathmoveto{\pgfqpoint{0\pgfutil@tempdima}{0\pgfutil@tempdima}}%
	\pgfpathmoveto{\pgfqpoint{-6pt}{-6pt}}% 
	\pgfpathcircle{\pgfpoint{-11.5pt}{0}} {3.5pt}
	\pgfusepathqstroke%
}

\pgfarrowsdeclare{one}{one}
{
	\pgfarrowsleftextend{+-.5\pgflinewidth}%
	\pgfarrowsrightextend{+.5\pgflinewidth}%
}
{
	\pgfutil@tempdima=0.6pT
	\pgfsetdash{}{+0pt}
	\pgfsetmiterjoin
	\pgfusepathqstroke
}
\makeatother
\tikzset{%
	pics/entity/.style n args={3}{code={%
			\node[draw,
			rectangle split,
			rectangle split parts=2,
			text height=1.5ex,
			] (#1)
			{#2 \nodepart{second}
				\begin{tabular}{>{\raggedright\arraybackslash}p{8.5em}}
					#3
				\end{tabular}
			};%
	}},
	pics/entitynoatt/.style n args={2}{code={%
			\node[draw,
			text height=1.5ex,
			] (#1)
			{#2};%
	}},
	zig zag to/.style={
		to path={(\tikztostart) -| ($(\tikztostart)!#1!(\tikztotarget)$) |- (\tikztotarget)}
	},
	zig zag to/.default=0.5,   
	one to one/.style={
		one-one, zig zag to
	},
	one to oone/.style={
		one-one, zig zag to
	},
	one to many/.style={
		one-crow's foot, zig zag to,
	},
	one to omany/.style={
		one-omany, zig zag to
	}
}

\makeatletter
\def\squarecorner#1{
	\pgf@x=\the\wd\pgfnodeparttextbox
	\pgfmathsetlength\pgf@xc{\pgfkeysvalueof{/pgf/inner xsep}}
	\advance\pgf@x by 2\pgf@xc
	\pgfmathsetlength\pgf@xb{\pgfkeysvalueof{/pgf/minimum width}}
	\ifdim\pgf@x<\pgf@xb
	\pgf@x=\pgf@xb
	\fi
	\pgf@y=\ht\pgfnodeparttextbox
	\advance\pgf@y by\dp\pgfnodeparttextbox
	\pgfmathsetlength\pgf@yc{\pgfkeysvalueof{/pgf/inner ysep}}
	\advance\pgf@y by 2\pgf@yc
	\pgfmathsetlength\pgf@yb{\pgfkeysvalueof{/pgf/minimum height}}
	\ifdim\pgf@y<\pgf@yb
	\pgf@y=\pgf@yb
	\fi
	\ifdim\pgf@x<\pgf@y
	\pgf@x=\pgf@y
	\else
	\pgf@y=\pgf@x
	\fi
	\pgf@x=#1.5\pgf@x
	\advance\pgf@x by.5\wd\pgfnodeparttextbox
	\pgfmathsetlength\pgf@xa{\pgfkeysvalueof{/pgf/outer xsep}}
	\advance\pgf@x by#1\pgf@xa
	\pgf@y=#1.5\pgf@y
	\advance\pgf@y by-.5\dp\pgfnodeparttextbox
	\advance\pgf@y by.5\ht\pgfnodeparttextbox
	\pgfmathsetlength\pgf@ya{\pgfkeysvalueof{/pgf/outer ysep}}
	\advance\pgf@y by#1\pgf@ya
}
\makeatother

\pgfdeclareshape{square}{
	\savedanchor\northeast{\squarecorner{}}
	\savedanchor\southwest{\squarecorner{-}}
	
	\foreach \x in {east,west} \foreach \y in {north,mid,base,south} {
		\inheritanchor[from=rectangle]{\y\space\x}
	}
	\foreach \x in {east,west,north,mid,base,south,center,text} {
		\inheritanchor[from=rectangle]{\x}
	}
	\inheritanchorborder[from=rectangle]
	\inheritbackgroundpath[from=rectangle]
}

\tikzstyle{startstop} = [rectangle, rounded corners, minimum width=3cm, minimum height=1cm, text centered, draw=black, fill=red!20]
\tikzstyle{io} = [trapezium, trapezium left angle=70, trapezium right angle=110, minimum width=3cm, minimum height=1cm, text centered, draw=black, fill=blue!20]
\tikzstyle{process} = [rectangle, minimum width=3cm, minimum height=1cm, text centered, draw=black, fill=orange!20]
\tikzstyle{decision} = [diamond, minimum width=3cm, minimum height=1cm, text centered, draw=black, fill=green!20]

\tikzstyle{boxNode} = [rectangle, rounded corners, minimum width=3cm, minimum height=1cm, text centered, draw=black]
\tikzstyle{conditionNode} = [diamond, minimum width=3cm, minimum height=1cm, text centered, draw=black]

\usepackage{subfiles} % Best loaded last in the preamble
