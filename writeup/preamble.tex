\usepackage{amsmath}
\usepackage{amssymb}
\usepackage{tikz}
\usepackage{pgfplots}
\usepackage{subfig}
\usepackage{physics}
\usepackage{algorithmicx}
\usepackage{algpseudocode}
\usepackage{listings}     
\usepackage{lstautogobble}  % Fix relative indenting
\usepackage{color}          % Code coloring
\usepackage{zi4}            % Nice font
\usepackage{hyperref}
\usepackage{yfonts}
\usepackage{hyperref}
\usepackage{xcolor}
\usepackage{graphicx}
\usepackage{lipsum}
\usepackage{adjustbox}
\usepackage{flushend}
\usepackage{booktabs}
\usepackage{siunitx}
\usepackage{blindtext}
\usepackage{multicol}
\usepackage[breakable,
			skins]{tcolorbox}
\usepackage[protrusion=true,
			activate={true,nocompatibility},
			final,
			tracking=true,
			kerning=true,
			spacing=true,
			factor=1100]{microtype}
\usepackage{fontawesome}

\usetikzlibrary{shapes, decorations, calc, arrows}
\usetikzlibrary{3d,fit,backgrounds, decorations.text}
\usetikzlibrary{positioning, shapes.symbols}
\usetikzlibrary{decorations.pathreplacing, calligraphy}

\pgfplotsset{width=10cm,compat=1.18}
\usepgfplotslibrary{external}

\tikzexternalize[prefix=cache/tikz]
\tikzset{>=latex}

\SetTracking{encoding={*}, shape=sc}{40}

\hypersetup{
	colorlinks,
	linkcolor={red!50!black},
	citecolor={blue!50!black},
	urlcolor={blue!80!black}
}

\lstdefinestyle{paperStyle1}{
	backgroundcolor=\color{backcolour},   
	commentstyle=\color{codegreen},
	keywordstyle=\color{magenta},
	numberstyle=\tiny\color{codegray},
	stringstyle=\color{codepurple},
	basicstyle=\ttfamily\footnotesize,
	breakatwhitespace=false,         
	breaklines=true,                 
	captionpos=b,                    
	keepspaces=true,                 
	numbers=left,                    
	numbersep=5pt,                  
	showspaces=false,                
	showstringspaces=false,
	showtabs=false,                  
	tabsize=2
}

\lstdefinestyle{paperStyle2}{
	autogobble,
	columns=fullflexible,
	showspaces=false,
	showtabs=false,
	breaklines=true,
	showstringspaces=false,
	breakatwhitespace=true,
	escapeinside={(*@}{@*)},
	identifierstyle=\color{pinkidentifiers},
	commentstyle=\color{greencomments},
	keywordstyle=\color{bluekeywords},
	stringstyle=\color{redstrings},
	numberstyle=\color{graynumbers},
	morekeywords={psrotate},
	basicstyle=\ttfamily\footnotesize,
	frame=l,
	numbers=left,
	framesep=22pt,
	xleftmargin=22pt,
	tabsize=2,
	captionpos=b
}

% \definecolor{bluekeywords}{rgb}{0.13, 0.13, 1}
% \definecolor{greencomments}{rgb}{0, 0.5, 0}
% \definecolor{redstrings}{rgb}{0.9, 0, 0}
% \definecolor{graynumbers}{rgb}{0.5, 0.5, 0.5}

\definecolor{pinkidentifiers}{rgb}{1, 0.53, 0.53}
\definecolor{bluekeywords}{rgb}{0.53, 0.53, 1}
\definecolor{greencomments}{rgb}{0, 0.5, 0}
\definecolor{redstrings}{rgb}{0.9, 0, 0}
\definecolor{graynumbers}{rgb}{0.5, 0.5, 0.5}

\setlength\columnsep{30pt} % This is the default columnsep for all pages

\lstset{style=paperStyle2}
\sisetup{load-configurations = abbreviations, binary-units = true}
\DeclareSIUnit\px{px}

% Rename "Contents" to "Summary"
\renewcommand*\contentsname{Summary}

% Register codeword as a command
\NewDocumentCommand{\codeword}{v}{%
	\texttt{\textcolor{darkgray}{#1}}%
}

\graphicspath{{images/}}

% Nice grey box
\newtcolorbox{greybox}[1]{colback=black!5!white,colframe=white!25!black,title={#1}}

% Red box
\newtcolorbox{redbox}[1]{colback=red!5!white,colframe=red!50!black,title={#1}}

% Blue box
\newtcolorbox{bluebox}[1]{colback=blue!5!white,colframe=blue!50!black,title={#1}}

% Green box
\newtcolorbox{greenbox}[1]{colback=green!5!white,colframe=green!50!black,title={#1}}

% Code box
\newtcolorbox{codebox}[1]{colback=blue!15!black,colframe=white!50!black,title={\faCode \quad #1}}

\usepackage{subfiles} % Best loaded last in the preamble
